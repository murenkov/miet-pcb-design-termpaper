\chapter{Библиотека компонентов}

Большую часть компонентов можно взять из библиотеки StudentsLibrary, но часть специфичных для проекта необходимо создать самостоятельно: малошумящие усилители, фазовращатели, использованные при моделировании ранее, а также микроконтроллер.

Все выбранные компоненты выполнены в QFN-корпусе, поэтому для их создания их посадочных мест можно использовать Wizard.

Покажем созданные УГО и посадочное место на примере микроконтроллера (Рис.~\ref{fig:GD32VF103}).
\begin{figure}[H]
    \centering
    \begin{subfigure}{0.48\textwidth}
        \centering
        \includegraphics[width=\textwidth]{GD32VF103-SDS.pdf}
        \caption{}%
        \label{fig:GD32VF103-SDS.pdf}
    \end{subfigure}
    \hfill
\begin{subfigure}{0.48\textwidth}
        \centering
        \includegraphics[width=\textwidth]{GD32VF103-footprint.pdf}
        \caption{}%
        \label{fig:GD32VF103-footprint.pdf}
    \end{subfigure}
    \caption{fig:GDF_micruxa}%
    \label{fig:GD32VF103}
\end{figure}
